% Template for ICIP-2018 paper; to be used with:
%          spconf.sty  - ICASSP/ICIP LaTeX style file, and
%          IEEEbib.bst - IEEE bibliography style file.
% --------------------------------------------------------------------------
\documentclass{article}
\usepackage{spconf,amsmath,graphicx}
\usepackage{epstopdf}
\usepackage{color}
\usepackage{enumerate}
\usepackage{amsmath}
\usepackage{booktabs}


%\newcommand{\cxj}[1]{\textcolor[rgb]{1.00,0.00,0.00}{(cxj: #1)}}
\newcommand{\mdf}[1]{\textcolor[rgb]{0.00,0.00,1.00}{#1}}
%\newcommand{\drf}[1]{\textcolor[rgb]{0.00,0.50,1.00}{(drf: #1)}}
\newcommand{\vb}[1]{\mathbf{#1}}
\newcommand{\comments}[1]{}
% Example definitions.
% --------------------
\def\x{{\mathbf x}}
\def\L{{\cal L}}

% Title.
% ------
\title{Robust Room Layout Estimation from a Single Image \\ with Geometric Hints}
%
% Single address.
% ---------------
\name{Ruifeng Deng, Xuejin Chen}
\address{University of Science and Technology of China }
%
% For example:
% ------------
%\address{School\\
%	Department\\
%	Address}
%
% Two addresses (uncomment and modify for two-address case).
% ----------------------------------------------------------
%\twoauthors
%  {A. Author-one, B. Author-two\sthanks{Thanks to XYZ agency for funding.}}
%	{School A-B\\
%	Department A-B\\
%	Address A-B}
%  {C. Author-three, D. Author-four\sthanks{The fourth author performed the work
%	while at ...}}
%	{School C-D\\
%	Department C-D\\
%	Address C-D}
%
\begin{document}
%\ninept
%
\maketitle
%
\begin{abstract}
%In this paper, we consider the problem of room layout estimation from a single RGB image. Previous methods based on fully convolutional neural networks (FCNs) suffer from heavy occlusion and clutter in indoor scenes. 
Estimation of room layout suffers from heavy occlusions and clutters in indoor scenes.
%
In this paper, we propose a deep network that combines textures and geometric hints to predict the surface layout from a single image.
%
Our method consists of three steps. First, depths and normals are extracted from the input RGB image. Secondly, a multi-channel FCN (MC-FCN) is presented to integrate these geometric hints for semantic surface segmentation. Thirdly, an optimization framework is adopted to refine the layout estimation. 
The results on two commonly used benchmark datasets demonstrate the robustness of our method on complex scenes. 
	
\end{abstract}
%
\begin{keywords}
Layout estimation, scene understanding, geometric hints
\end{keywords}
%
\section{Introduction}
\label{sec:intro}

The main purpose of room layout estimation is to extract semantic boundaries among walls, ceiling and floor from a single image. It can be realized by obtaining corresponding semantic planes, as shown in Fig. \ref{fig:pipeline}. The task is challenging because of complex environmental factors such as illumination variation, viewpoint shift, object diversity and clutter. Furthermore, distinctive clues like room corners and layout boundaries are often occluded by objects. 

%%
In recent years, massive researches have been carried out on room layout estimation. Conventional methods usually follow a proposing-ranking framework \cite{hedau2009recovering,wang2013discriminative,gupta2010estimating,hedau2010thinking}. Typically, they first generate numbers of proposals through vanishing point detection and ray sampling. Then a ranking step using hand-crafted rules is adopted to select the best hypothesis. Recent methods built on FCNs \cite{long2015fully} or encoder-decoder networks achieve impressive performance \cite{mallya2015learning,ren2016coarse,zhang2017learning,dasgupta2016delay,LeeRoomNet17,zhao2017physics}. Specifically, 
%
Mallya et al.~\cite{mallya2015learning} presented an FCN for learning informative edge maps from images, which were then integrated into the conventional framework as additional features. 
%
Dasgupta et al.~\cite{dasgupta2016delay} used an FCN to learn semantic surface labels including \{Left wall, Front wall, Right wall, Ceiling, Ground\} for each pixel. They obtained a segmentation depended on heat maps from FCN and further refined it with geometric projection constraints.
% 
In a subsequent technique~\cite{ren2016coarse}, Ren et al. adopted a multi-task fully convolutional neural network to jointly predict the surface labels and boundaries. 
%
Zhang et al.~\cite{zhang2017learning} proposed a deconvolution network which has multi-layer deconvolution to attain highly reliable edge maps. 
%
Lee et al.~\cite{LeeRoomNet17} explored an end-to-end approach for room layout estimation. They employed an encoder-decoder network to delineate the room layout structure using 2D keypoints.
%
Zhao et al. \cite{zhao2017physics} introduced a semantic transfer FCN to extract reliable edge features and designed a physics inspired inference scheme for optimization.


Different from previous FCN based methods, we propose to improve the performance of room layout estimation from another perspective. Inspired by some work on RGBD semantic segmentation \cite{gupta2014learning,couprie2013indoor,li2017semantics,pagnutti2017segmentation,qi20173d}, we consider global geometric information as important as textures in room layout estimation. To implicitly employ the geometric hint, depth and normals are estimated from the original image and fused into an multi-channel FCN. We use the same formulation of layout in \cite{dasgupta2016delay} and combine their optimization step with a proposing-ranking framework to attain precise layout. Experimental results show that our method is robust and effective for layout estimation even facing clutter and occlusion.

\comments{
In this paper, we propose to enhance the FCN's ability for room layout estimation by fully utilizing the geometric information of the scene. Our pipeline is shown in Fig.~\ref{fig:pipeline}. First, we estimate depth and normal maps from a single image using network in \cite{eigen2015predicting}. Then we integrate these geometric hint to train a multi-channel FCN for semantic surface segmentation. At last, an extended post-processing algorithm based on \cite{dasgupta2016delay} is applied to generally refine the estimation results. We evaluate our method on two popular benchmark dataset. Experimental results demonstrate that our method is robust and effective for layout estimation even facing clutter and occlusion.
}






\section{Our Method}
\label{sec:Meth}


\subsection{System Overview}
\label{subsection:overview}

\begin{figure}[!ht]
	\centering
	\includegraphics[width=3.4in]{figure/pipeline.eps}
	\caption{An overview of our layout estimation algorithm pipeline. First we adopt a multi-scale CNN achitecture \cite{eigen2015predicting} to extract geometric information from RGB images, including depth and normals. Then we combine all the abovementioned information into a multi-channel FCN , which helps to accurately estimate the layout. An optimization framwork based on perspective projection restriction is then used to generate final precise layout estimates.}
	\label{fig:pipeline}
\end{figure}

Under the Manhattan world assumption, a room layout can be represented as a cube having at most five surfaces (Three walls, Ceiling, Ground) visible in the image. 
%
Given an RGB image $\vb{I}$ with arbitrary size, our algorithm generates a room layout $\vb{L}$ consisting of a surface label for each pixel $L_{ij}\in $ $\{$Left, Front, Right, Ceiling, Ground$\}$. 
Fig. \ref{fig:pipeline} shows our algorithm pipeline. 
%%step 1
Different from \cite{dasgupta2016delay}, we first estimate depth map $D_{I}$ and normal map $N_{I}$ from the input color image to generate geometric hints using a multi-scale convolutional architecture~\cite{eigen2015predicting}, as described in Sec.~\ref{sec:depth_normal}.
%step 2
After that, to integrate information from the original RGB image together with the geometric hints, a fully convolutional network with multi-channel input is applied to predict five probability maps, each of which describes the belief for a specific layout surface, see details in Sec.~\ref{sec:surfacelabel}.
%step 3: optimization
While we can easily get access to the final layout estimation by just choose the label with highest score across the five probability maps for each pixel. 
The estimated $\vb{L}$ always have wavy boundaries and multiple disjoint connected domains due to the characteristics of neural network. To handle these problems and obtain more clear layout estimation with geometric constraints, an optimization step is adopted in Sec.~\ref{sec:optimization}.  
 

\comments{
The coarse layout estimation about semantic layout surfaces for a single RGB image, are predicted by fully convolutional neural network with geometric information emmbeded, including depth and normal information. These geometric information are estimated from source RGB image by a multi-scale convolutional architecture\cite{eigen2015predicting}. This will be decribied in Sec. \ref{subsection:CNN}. Then based on optimization framwork proposed in \cite{dasgupta2016delay}, which mainly uses perspective projection constraints, we can obtain final precise layout estimation results.
%
}


\subsection{Geometric Hints Extraction}
\label{sec:depth_normal}

We use the multi-scale convolutional network proposed in \cite{eigen2015predicting} to estimate the depth and normal map from a single RGB image, as shown in Figure~\ref{fig:depthandnormal}. Note that we do not finetune their models with our data as we do not have the depth and normals of our dataset, however, the model seems to generalize well in our case. From the global perspective, the depth and normal maps we obtained capture the overall spatial relationships between semantic surfaces. Quantitive evaluation for depth and normals extraction can be refered to in ~\cite{eigen2015predicting}.
%

\begin{figure}
	\centering
	\includegraphics[width=\columnwidth]{figure/geometricinfo1.png}
	\caption{Estimation of depth and normal from a single RGB image using the multi-task FCN in \cite{eigen2015predicting}.}
	\label{fig:depthandnormal}
\end{figure}

We notice in practice that the depth maps and surface maps estimated from single RGB images are not precise in local details. And the The output resolution of the depth maps and normal maps are limited to half of the input. However, the valuable 3D information they provide is much helpful for high-level structure estimation, especially in cluttered scenes.  
%
Normals can serve as clues tending to merge big planes together, with interference factors like clutter, textures and illumination eliminated to varying degrees. 
%
These mid-level geometric information can be used to adapt and improve performance for layout estimation compared to using RGB only. 

The depth maps are color coded in \cite{eigen2015predicting} for distinction. In order to reduce the redundant information and to ensure the essence of depth, We modify their rendering section to obtain depth maps with single channel while preserving the optimization part.

\subsection{Surface Label Prediction Using MC-FCN}
\label{sec:surfacelabel}
%
In previous work, fully convolutional neural network(FCNN) or multi-task fully convolutional neural network~(MFCNN) are used to predict coarse semantic layout surfaces or layout edges \cite{dasgupta2016delay,ren2016coarse,zhao2017physics,zhang2017learning}. 
%
However, due to much clutter, complex textures and illumination variations, semantic surfaces like walls are visually separated in to pieces, making whole surfaces difficult to aggregate together. 
%
Fig.~\ref{fig:fcn-comparison}(b) shows some bad semantic surface predicting results from a widely used FCNN architecture in previous methods \cite{dasgupta2016delay,ren2016coarse}. 
The comparison between the predicted results and the ground truth reveals that due to the complex environmental factors of indoor scenes, layout results estimated from FCNN are not very reliable sometimes. 
When the occlusion happens to be at the boundaries of semantic planes, such critical clues are partially or entirely excluded, and thus we can not tell location of each plane precisely. 
Moreover, an entire plane may be predicted separately into pieces due to clutter lay in the plane.


\begin{figure}
	\centering
	\includegraphics[width=\columnwidth]{figure/fcn-multi-channel.png}
	\caption{Illustration of the multi-channel network architecture build on \cite{chen2016deeplab}. We combine the RGB image, depth and normal together as input to the ResNet101 based FCN for semantic surface learning. }
	\label{fig:fcn-multi-channel}
\end{figure}


\textbf{Network Architecture}
To alleviate the problems caused by complex environmental factors, we fuse the geometric hints with original texture information and train a multi-channel input FCN. As Fig.~\ref{fig:fcn-multi-channel} shows, We build this multi-channel FCN on the famous architecture ResNet-101 \cite{he2016deep} modified by \cite{chen2016deeplab}. The estimated depth and normal maps are treated as additional channels associated with the input RGB images. 

\textbf{Training}
We simply concatenate different types of data to a seven-channels input(3 channels from RGB, 1 channel from depths and 3 channels from normals) which is then fed into our network. The FCN model we used for initialization from \cite{chen2016deeplab} is pretrained on MS-COCO dataset. We randomly initialize the weights for the first convolution layer to adapt to our input format and replace the weights for classifier layer with randomly initialized 5-way classifier layer. Then we specifically finetune our multi-channel FCN for semantic surface segmentation. We have also tried multi-task learning by adding a new branch for learning boundaries between surfaces as \cite{ren2016coarse, mallya2015learning} do. But the joint training doesn't help to improve the performance of both task in our case. This may indicate that our geometric hints have already integrated the additional information which should be learned through joint training.

The output of our multi-channel FCN is a $w\times h \times 5$ multidimensional array $\vb{T}$, where w and h stand for the width and length of the input rgb image. Each of the 5 slices can be viewd as a probability map for the corresponding surface label. Some visualized results compared with FCN without geometric hints as additional input are shown in Fig.~\ref{fig:fcn-comparison}. We utilize the probability array $\vb{T}$ as the basis for our scoring function. Then an optimization step based on this score function is applied to obtain precise layout estimation.


\begin{figure}[!ht]
	\centering 
	\textsc{\includegraphics[width=3.4in]{figure/fcn.eps}}
	\caption{Layout estimation results using different architectures. Row from top to bottom: (a) the input RGB image. (b) Surface predictions using the FCNN architecture used in \cite{dasgupta2016delay,ren2016coarse}. (c) our MC-FCN. (d) The ground truth. Our method generate much more clear edges, less holes.}
	\label{fig:fcn-comparison}
\end{figure}


\subsection{Layout Generation}
\label{sec:optimization}
We adopt a popular model that several researchers ~\cite{hedau2009recovering, wang2013discriminative, dasgupta2016delay, ren2016coarse} have used to parameterize indoor layout based on the Manhattan world assumption. 
Indoor scene layout is modeled as the projection of a cuboid which can be defined by 

\begin{equation}
	\label{eq:Layout}
	L = (l1, l2, l3, l4, v)
\end{equation}

where $l_{i}$ stands for $i^{th}$ line and $v$ stands for the specific vanishing point. The whole scene is equivalent to be labeled with five semantic surfaces, coresponding to \{Left, Front, Right, Ceiling, Ground\}, as described in Fig. \ref{fig:2.model}. Based on Eq. (\ref{eq:Layout}), each surface can be reconstructed with extension lines between vanishing point $v$ and Intersection point $p_i$ among $l_{i}$. ie $ve_i$ . Due to the uncertainty of camera pose, five surfaces are not always visible. Examples with different topologies are given in Fig. \ref{fig:different-layout-type}.


To obtain the optimized box layout $\vb{L}$, \cite{dasgupta2016delay} has propose a refinement algorithm. They first generate an initial $\vb{\hat{L}}$ by pick the label with the highest score across five channels of $\vb{T}$. Then a preprocessing step is used to address spurious regions and multiple disjoint components in $\vb{\hat{L}}$. Finally, an iterative optimization method is employed to acquire the refined $\vb{L}$. However, the initial $\vb{\hat{L}}$ generated from $\vb{T}$ are sometimes too ambiguous due to the characteristics of neural network. The preprocessing step can't adapt to all the bad situations well, especially when the ambiguous walls are close in size. 


To make the optimization method more general and efficient, we simplify the refinement algorithm in \cite{dasgupta2016delay} by replace the first two step with a proposing-ranking framework. Our proposing-ranking framework directly generate an initial $\vb{\hat{L}}$ that consistent with the definition in Eq.~\ref{eq:Layout}. This kind of $\vb{\hat{L}}$ are much robust to ambiguous cases and easier for the optimization step to find the refined layout.

The proposing-ranking framework is implemented by extracting enough proposals using the approach in \cite{hedau2009recovering} and a score function based on $\vb{T}$. We sample 30 rays per vanishing point to acquire enough proposals. Then we extend the modified score function in \cite{dasgupta2016delay} for special cases to adapt to all images. Given a specific label, We simply sum up the probability across 5 channels of $\vb{T}$ seperately in the same region and choose the maximum as the score for this specific label. This policy is applied just among three walls which mainly cause the ambiguity. The same extended score function is adopted in optimization step. Our postprocessing framework generalize well to ambiguous cases.

\comments{
\textbf{Preprocessing} 
Although we have more robust surface prediction result from our FCN-MC. Problems like spurious regions and multiple disjoint components are still a bit disturbing. To address these problems we simplify the preprocessing step in ~\cite{dasgupta2016delay} for efficiency. First, we address the multiple disjoint components (which means the connection domain for each label is not unique) by remove all but the largest connected domain for each label. Then we fill the holes created by this pruning with labels around them by using k-nn. After that we apply some simple contraints based on common sense to handle spurious regions.}

\begin{figure}
	\centering
	\includegraphics[width=\columnwidth]{figure/different-layout-type.png}
	\caption{Different layout type due to variation in camera pose. \drf{I'm not sure whether this situation should be discussed, so the caption is brief.} }
	\label{fig:different-layout-type}
\end{figure}
\section{RESULTS}
\label{sec:Res}

We evaluate our method on two widely used dataset: Hedau's dataset \cite{hedau2009recovering} and LSUN dataset.
\subsection{LSUN Results}
\label{sec:LSUN}
We train our multi-channel FCN on the relabeled LSUN dataset released by \cite{ren2016coarse}. The dataset consists of 4000 training, 394 validation, and 1000 testing images. We extract geometric hints from original images and resize all the images, depth and normal maps to $321\times321$ using bicubic interpolation. Then these three types of data are integrated to train the ResNet101 based multi-channel FCN. We evaluate our results using the official toolkit which provide two standard metrics: pixelwise error and corner error. The pixelwise error is computed by counting the percentage of pixels that are mismatched. (The Hungarian algorithm is applied to address the labeling ambiguity problem) The corner error is computed by calculate the Euclidean distance between predicted corners and corresponding ground truth corners.

Our performance on LSUN test set compared with other methods is shown in Table~\ref{table:comparison-lsun}. ... 

\begin{table}
	\centering 
	\caption{Performance benchmarking on the LSUN dataset}
	\label{table:comparison-lsun}
	\begin{tabular}{l|c|c}
		\hline 
		Method & Corner Error (\%) & Pixel Error (\%) \\
		\hline
		Hedau et al. (2009)~\cite{hedau2009recovering} & 15.48 & 24.23 \\
		Mallya et al. (2015)~\cite{mallya2015learning} & 11.02 & 16.71 \\
		Zhang et al. (2017)~\cite{zhang2017learning} & 8.70 & 12.49 \\
		Dasgupta et al. (2016)~\cite{dasgupta2016delay} & 8.20 & 10.63 \\
		Ren et al. (2016)~\cite{ren2016three} & 7.95 & 9.31 \\
		Proposed FCN-GF & xxx & 9.51 \\
		\hline
	\end{tabular}
\end{table}

\subsection{Hedau Results}
\label{sec:Hedau}
We also conduct experiment on a relatively smaller dataset published by Hedau \emph{et al}. \cite{hedau2009recovering}, which consists of 209 training images and 104 testing images. We directly predict the semantic surface on Hedau's test set using the model trained on LSUN. Pixel error is adopted to evaluate the results, see Table~\ref{table:comparison-hedau}. ...


\begin{table}
	\centering 
	\caption{Performance benchmarking on Hedau's dataset}
	\label{table:comparison-hedau}
	\begin{tabular}{l|c}
		\hline 
		Method & Pixel Error (\%) \\
		\hline
		Hedau et al. (2009)~\cite{hedau2009recovering} & 21.20 \\
		Mallya et al. (2015)~\cite{mallya2015learning} & 12.83 \\
		Zhang et al. (2017)~\cite{zhang2017learning} & 12.7 \\
		Dasgupta et al. (2016)~\cite{dasgupta2016delay} & 9.73 \\
		Ren et al. (2016)~\cite{ren2016three} & 8.67 \\
		Zhao et al. (2017)~\cite{zhao2017physics} & 6.60 \\
		Proposed MC-FCN & xxx \\
		\hline
	\end{tabular}
\end{table}


\subsection{Analysis of Geometric Hints}
\label{sec:ablation}
To explore the benefits of geometric hints for semantic surface segmentation, we train different FCNs with or without geometric information as additional input. And in order to compare with \cite{ren2016coarse}, we use the FCN architecture based on VGG16. Qualitative results on LSUN validation set are demonstrated by Fig.~\ref{fig:fcn-comparison}. Intuitively, the geometric hints help to remove the spurious regions caused by clutter and generate more accurate semantic segmentation. Quantitive results are shown in Table~\ref{table:ablation}. 

\begin{table}
	\centering
	\caption{Pixelwise accuracy for surface label prediction.}
	\label{table:ablation}
	\begin{tabular}{c|c}
		\hline
		Network & Accuracy\\
		\hline
		FCN-32s & 0.8109 \\ 
		MC-FCN  & 0.8392 \\
		Ren et al. (2016)~\cite{ren2016three} & xxx \\
		\hline
	\end{tabular}
	
\end{table}



\section{CONCLUSION}
\label{sec:Con}

In this paper, we propose to fully utilize the geometric information from an RGB for room layout estimation. We estimate the depth and normal maps from the original image and then combine them together to train a multi-channel FCN. We demonstrate how these geometric hints can help to improve the performance of semantic surface segmentation. We further extend the optimization framework by incorporating a proposing-ranking policy. Experiments have been conducted on two benchmark datasets. ...


% Below is an example of how to insert images. Delete the ``\vspace'' line,
% uncomment the preceding line ``\centerline...'' and replace ``imageX.ps''
% with a suitable PostScript file name.
% -------------------------------------------------------------------------
 

% To start a new column (but not a new page) and help balance the last-page
% column length use \vfill\pagebreak.
% -------------------------------------------------------------------------
%\vfill
 
% References should be produced using the bibtex program from suitable
% BiBTeX files (here: strings, refs, manuals). The IEEEbib.bst bibliography
% style file from IEEE produces unsorted bibliography list.
% -------------------------------------------------------------------------
\bibliographystyle{IEEEbib}
\bibliography{refs}

\end{document}
