\section{RELATEDWORK}
\label{sec:Rel}

%conventional method
Layout estimation of indoor scenes has drawn substantial attention in computer vision fields since Hedau \emph{et al}. \cite{hedau2009recovering} formulated it as a structured learning problem. They introduced a three-step pipeline under famous Manhattan assumption \cite{coughlan1999manhattan}. First, vanishing points were estimated from detected line segments in a voting way. Then, a lot of layout hypotheses were created by ray sampling from the vanishing points. Finally, they trained a structured regressor using hand-crafted features to evaluate hypotheses. Later on, many conventional studies were carried out based on this milestone framework \cite{gupta2010estimating, wang2013discriminative,hedau2010thinking,schwing2013box,ramalingam2013manhattan}. They proposed different ranking criteria to remove bad hypotheses or incorporated new features to improve the ranking accuracy. 

%fcn based methods
Motivated by the impressive progresses in semantic segmentation using fully convolutional neural networks \cite{long2015fully,chen2016deeplab}, many researchers tend to utilize this powerful deep learning model for room layout estimation. 
%
Mallya \emph{et al}. \cite{mallya2015learning} presented an FCN for learning informative edge maps from images, which were then integrated into the conventional framework as additional features. 
%
Dasgupta \emph{et al}. \cite{dasgupta2016delay} used an FCN to learn semantic surface labels including \{Left wall, Front wall, Right wall, Ceiling, Ground\} for each pixel. They obtained a segmentation depended on heat maps from FCN and further refined it with geometric projection constraints.
% 
In a subsequent work \cite{ren2016coarse}, Ren \emph{et al}. adopted a multi-task fully convolutional neural network (MFCN) to jointly predict the surface labels and boundaries. Benefit from joint training, the semantic boundaries were more robust and accurate. They also optimized the results later with a refinement framework designed for boundaries.
% 
Zhang \emph{et al}. \cite{zhang2016learning} proposed an another deconvolution network which has multi-layer deconvolution and a receptive field as large as the entire image compared to FCN. As a result, they attained highly reliable edge maps. Then they extended the hypotheses generation method in \cite{mallya2015learning} with an adaptive line sample strategy. However, their final results were less precise compared with \cite{dasgupta2016delay,ren2016coarse}.
%
Zhao \emph{et al}. \cite{zhao2017physics} introduced a semantic transfer FCN to extract reliable edge features. They fisrt trained an FCN for 37-class semantic segmentation and then bridge the gap to 4-class edge labels by adding a fully connected layer. A physics inspired inference scheme was designed for opimization. 


Different from these FCN based method, we propose to improve the performance of room layout estimation from another perspective. We consider that global geometric information is as important as texture information in room layout estimation. To implicitly employ the geometric information, we estimate depth an normals from the original image. We use the same formulation of layout in \cite{dasgupta2016delay} and combine their optimization step with a proposing-ranking framework to attain precise results in the form of  Fig.~\ref{fig:definition}(b)(c).  
